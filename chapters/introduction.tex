\chapter{Introduction}\label{ch:introduction}
This deliverable presents the methodology and tools for the collaborative development of the pilots of the Polifonia project. 
It describes the scope and role of the Technical Board, established for coordinating the technical development and ensure the quality and consistency of the technical outputs. 
The first objective of the Technical Board was to define a collaboration methodology, following the spirit of agile software engineering methodologies, which combines a two-fold approach. 
On the one hand, developers and domain experts are joined in sessions dedicated to specific research themes, and develop ideas on how to approach specific problems, at the micro level. 
On the other hand, Work Packages and Task leaders participate in those sessions with the aim of developing connections to the objectives of the project, at the macro level. 
A variety of different types of assets are expected to be produced by the co-creation process. 
Those include not only software libraries, data, services, and user interfaces but also information to make sense of them: stories, representing requirements in the forms of scenarios, tutorials, and how-to. 
The components of the Polifonia Ecosystem are designed as interlinked assets for supporting the development of innovative tools for musical cultural heritage study, preservation, and exploitation. 
These components constitute the technical backbone of the pilots, which combine them in useful applications targeting scholars and citizens. 
Developers will pick and mix components of the ecosystem and compose them into complex pipelines, to build interesting applications. 
However, the Polifonia Ecosystem does not start as a blank sheet. 
Consortium members bring expertise and technical solutions that can already be employed in such modular approach. % Therefore, background technologies relevant to the ecosystem are presented. 
The project has already implemented key actions to bootstrap the development activities, including dedicated spaces on a collaborative development platform (GitHub) and a live text chat system (Discord). 
The methodology and tools are already employed by the working groups focusing on the development of the Polifonia Pilots, that we also summarise in this deliverable. 
Finally, we report on ongoing work towards a Polifonia Web Portal, an aggregator of Musical Heritage Knowledge.


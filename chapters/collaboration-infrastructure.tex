\chapter{Collaboration Infrastructure}\label{ch:infrastructure}
The Technical Board setup the following infrastructure to support the collaborative development of the Pilots.

\textbf{GitHub} is the leading code hosting platform supporting development teams in tasks such as version control and collaboration. The TB established the \textbf{polifonia-project} GitHub organisation\footnote{\url{http://github.com/polifonia-project}} as the collaboration space for the technical activities of the consortium. The GitHub organisation currently includes 30 registered GitHub user accounts belonging to contributing project members.
So far, the collaborative space was used to collect stories and scenarios, including managing the discussion and actions of activity streams emerged in the Maninpasta workshop. Repositories include: \textbf{stories} -- which collects requirements and issues emerging during the first two Maninpasta workshops, \textbf{registry} -- which is dedicated to the development of a Polifonia Registry of musical resources, from the original work of~\cite{daquino2017characterizing}, and \textbf{ecosystem} -- aimed at hosting the Polifonia Ecosystem Website.

\textbf{Discord} is a platform supporting invite-only social media. The TB created a Polifonia Server that is currently used by project members for fast interaction. Communication is organised in channels:
\begin{itemize}
    \item general -- for general discussions
    \item dev -- for technical discussions of developers and the technical board
    \item maninpasta -- for live interaction during the workshops
    \item wp1, wp2, wp3, wp4, wp5 -- for discussion relevant to specific WPs and for live interaction during periodic update meetings
\end{itemize}
Consortium members, task leaders, and working groups are free to create channels dedicated to specific activity streams. Currently, users created the following: \textit{sethusmockup}, \textit{textcorpus}, and \textit{ontologyengineering}.

Finally the infrastructure offered by the coordinator (UNIBO) -- via Microsoft Sharepoint -- includes a Technical Board \textbf{mailing list} that is open to all project members interested in following or contributing to technical activities.
Developers are particularly invited to subscribe, join technical board meetings, and use the list for technical discussions.

% Timeline 

% Deliverable at M6 

% Methodology -> coherent with Socio-technical roadmap 

% Guidelines for developers (cookbook) -> coherent with DMP 

% The Ecosystem at a glance (component types and interaction methods) 

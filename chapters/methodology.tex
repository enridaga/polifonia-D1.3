\chapter{Methodology}\label{ch:methodology}

The methodology used for the development of the pilots and the Polifonia ecosystem is inspired by agile software development methodologies \cite{collier2012agile}. Similar to those, our methodology focuses on ``discovering requirements and developing solutions through the collaborative effort of self-organising and cross-functional teams and their customer(s)/end user(s)''; in Polifonia, these cross-functional teams are organised by the various technology-oriented WPs (WP2 --knowledge graphs--, WP3 --music pattern discovery--, WP4 --text pattern discovery-- and WP5 --user interfaces--); while the customers/end users are represented by the various Pilots (WP1). Therefore, WPs and Pilots interact in an orthogonal way.

This distributed, self-organised approach has several advantages, especially when confronted with a more traditional, waterfall-based model \cite{benington1983production}:

\begin{itemize}
    \item Does not enforce specific non-functional requirements to any of the development teams, such as programming languages, libraries, or frameworks; therefore minimising the chances of writing large, monolithic systems that are hard to document and maintain (especially after the end of the project)\todo{@all: add something about long-term sustainability? E.g. innovation task force in WP6}
    \item Identifies and reduces dependencies between different Pilots
    \item It is feature-driven, and puts the requirements from the Pilots at the forefront of the development process
    \item Allows for incremental and frequent software releases under the open-source mantra ``release early, release often''
    \item Increases decoupling and independence of components, since each component has autonomous value and can be used by and combined with other components (e.g. the Pilots and the Web Portal)
    \item Increases opportunity for reuse of technical and scientific artefacts
    \item Allows for a decentralised quality assurance process, where project-global metrics can be implemented in component-specific ways
\end{itemize}

In addition to these well-known benefits, we extend the methodology with the following procedures:

\begin{enumerate}
    \item \textbf{Sign-off release}. The release of new software components into the ecosystem must follow a code review, involving one or two code reviewers from a different team from the one that carried out the development. More specifically, this happens when a \emph{development branch} requests a \emph{merge with its main branch} via a \emph{pull request}; such a pull request must include a code review request to the aforementioned external reviewers \todo{@all: check this is how we want to do it?}
    \item \textbf{Component development life-cycle}.
    \item \textbf{Bottom-up approach}. Component development, and especially the gathering, documentation and maintenance of requirements for such components, are managed through the Polifonia \emph{maninpasta} sessions. This is a hackathon-like, grassroots approach that periodically ensures interaction between requirement providers in the Pilots, and software development teams in the technological WPs.
    \item \textbf{Top-down approach}. WP leaders and TB members ensure that the bottom-up approach is aligned and converges, to the extent possible, to the goals established by Polifonia GA. This includes the establishment of WP checkpoints (e.g. previous to a milestone or a deliverable); and explicit breakdowns and plannings of how features, commits, releases, etc. align with the planning of WPs and their tasks. The TB synchronises and complements this top-down upservision in conjunction with the WP and Task leaders.
\end{enumerate}

\todo{Outline}
WP and Pilots are Orthogonal 
Goal: identify and reduce dependencies 
Increase opportunity for reuse (scientific / technical) 

Agile, feature-driven 

Incremental releases: "release early, release often” 

Decoupling / independence (each component has an autonomous value but *can* also be used with others, as exemplified in Pilots and Web Portal) 

Quality assurance 

Sign-Off release process, involving one or two code reviewers from another team 

Component development lifecycle 

Methodology Two-fold: 
(a) Bottom-Up: Maninpasta / hackatons grassroots approach 
(b) Top-Down: Checkpoint during WP, relation with WPs, tasks, etc … / Checkpoint on TB 
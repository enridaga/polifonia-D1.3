\chapter{The Polifonia Ecosystem}\label{ch:ecosystem}

The Polifonia Ecosystem is conceived as a collection of components which are both independent -- they have some value on their own -- and interlinked -- they can be combined in order to satisfy specific end-user needs.

% \section{Components}
Data components types are the following:
\begin{itemize}
\item \textbf{Registries} -- indexes of resources of interest to Musical Cultural Heritage. A preliminary example is the MusoW catalogue of Musical Resources on the Web\footnote{\url{musow.kmi.open.ac.uk}}. Other registries can be developed to fit specific needs (for example, the catalogue of resources useful to the CHILD pilot)
\item \textbf{Ontologies} -- produced in the context of the Polifonia project to support pilots and use cases, ontologies specify domain knowledge and are used as means for developing a shared understanding of the domain and as software artefacts applied in published datasets
\item \textbf{Datasets} -- structured data offered following best practices in (Linked) Open Data publishing. If data was not produced as Linked Data, when possible, the resource is also published in its original format. Multiple ontologies can be applied and alternative Linked Data versions of the same source data are possible, to fit the needs of different use cases.
\item \textbf{Repositories and corpora} -- collections of digital assets relevant to Polifonia use cases.
\item \textbf{Knowledge Graph} -- a distributed but unifying view of musical cultural heritage knowledge, is a \textit{virtual} composition of Linked Data resources to be reused for large scale integration, for example, to support unified indexes for exploration and discovery. %\todo[inline,author=Enrico]{We say that there will be 1 single KG aggregating everything and that the nature of the KG should be distributed and there can be one system which indexes the data to support some use cases such as discovery and reuse.}
\item \textbf{Services} -- Web APIs that expose reasoning and data processing capabilities. Services are run by Polifonia consortium members and instantiate specific components to the Open Web. % “Towards an open Web experience, not a “Facebook” experience” 
Among those there are Linked Data services such as SPARQL endpoints -- live data services publishing the above components for querying with SPARQL.
\item \textbf{Software libraries} -- reusable code produced by the project to support pilot activities.Software libraries are used by programmers in their own applications
\item \textbf{CLI tools} -- ready-made tools to be used by developers in scripting data manipulation pipelines
\item \textbf{User interfaces} -- targeting domain experts, citizens, developed to support specific activities in the context of the Polifonia Pilots, user interfaces can be reused across similar applications targeting different datasets and scenarios
\item \textbf{Experiments} -- code and dataset of scientific experiments used during the research activity of the project. Experiments are meant to be documented, reproducible, and linked to research outputs. However, experiments are not expected to produce code and data which is directly reusable. The developers may produce derived assets as independent components of the ecosystem
\item \textbf{Applications} -- applications targeting specific use cases, possibly as direct outputs of the Pilots. Applications may reuse ecosystem components and function as demonstrators in tutorials and referenced by the documentation of the ecosystem.
\item \textbf{Containers} -- which wrap applications or services ready to be deployed in a computing infrastructure (a developer's laptop or a cloud service).
% \item User Interface components
\item \textbf{Stories} -- requirements from the world of Musical Cultural Heritage preservation, exploitation, and scholarship; stories are the starting point of the collaborative methodology and the \textit{sense-making} layer of the Polifonia Ecosystemm, giving context and purpose to the components. Stories and scenarios are linked to relevant components of the ecosystem.
\item \textbf{Tutorials} -- a showcase of the Polifonia Ecosystem through end-to-end tutorials, inspired from the Pilots and displaying the capabilities of the components in concrete applications. Tutorials are also an excellent starting point for developers.
\item \textbf{Web Portal} (a KG view on Polifonia Linked Data resources) -- the aggregator of the Polifonia Knowledge, exploiting the Knowledge Graph as underlying integration method.
\item \textbf{Documentation} -- documentation associated to each component. Developed autonomously, to fit the requirements and needs of the specific type of component.
\item \textbf{Polifonia Ecosystem Website} (GitHub) -- The resource for accessing the Polifonia Ecosystem, browsing the components and associated the documentation, accessing the resources for developers, and joining the project team in building the next generation of tools for Musical Cultural Heritage.
\end{itemize}
%
% \todo[inline]{IMAGE: Polifonia Ecosystem}

One objective of the technical board is to identify issues preventing the interoperability of components and provide guidance on how to overcome them. Therefore, these component types are meant to be \textit{interlinked}. Inter-operability strategies are considered based on W3C and other industry standards for the development of distributed systems on the Web.
These include: W3C specifications such as RDF, SPARQL protocol, Linked Data Platform and the Solid project but also OpenAPI, JSON, JSON-LD and related technologies.
For a survey on  Web Technologies and their application for metadata aggregation in cultural heritage, we refer the reader to~\cite{freire2017web,daga2021integrating}.
%In this section we illustrate technologies that may contribute to the ecosystem. First, we identify technologies that can contribute to the development of distributed infrastructures for cultural heritage, taking the Web as solution space and referring to recent research in the area~\cite{}.
However, the Polifonia Ecosystem does not start as a blank sheet. 
Consortium members bring expertise and technical solutions that can already be employed in such modular approach.
Table~\ref{tab:technologies} list technologies that are part of the background of consortium members and that may contribute towards supporting Polifonia applications.

%Interoperation strategies: Web technologies (OpenAPI, SPARQL, Solid, URLs, …) 
% Registry 
% Datasets:
% - Ontologies, published following best practices 
% - SPARQL endpoints 
% Publishing guidelines
% - Life-cycle, Versioned Releases (on GitHub) 
% Software libraries:
% - Code quality 
% - Documentation  
% - Tutorials 
% - CLI 
%
% Services / Web APIs: “Towards an open Web experience, not a “Facebook” experience” 
% 
% - Availability 
% - Good practices (e.g. OpenAPI specification) 
% - Documentation 
% - Tutorials
% User Interface components:
% - Good practices: (a) Linkable UI views; (b) Reusable snippets (Web Embed) 
% - Reusability of UI 
% Deployment strategies / Interoperability How tos / Demonstrators
% - Recommendations and examples on how to reuse the components 
% - Recommendations on how to setup a docker components 
% 
% Stories and Scenarios (linked to pipelines / demos / relevant components) 
% Web Portal (a KG view on Polifonia) 
% Polifonia Ecosystem Documentation Website (GitHub) 
% Publishing Polifonia Components: metadata

\chapter{Contributions to the ecosystem}\label{ch:background}
\todo[inline]{Introductory paragraph}
\begin{table}[]
    \tiny
    \centering
    \caption{Technologies of consortium members that contribute to the Polifonia Ecosystem}
    \label{tab:technologies}
\begin{tabular}{|l|p{4cm}|c|p{2cm}|p{4cm}|}\hline\footnotesize
\textbf{Name} & \textbf{Links} & \textbf{Type} & \textbf{Champion} & \textbf{Relations to WPs / Pilots / Notes} \\\hline
LED & http://led.kmi.open.ac.uk 

Data available at http://data.open.ac.uk/sparql & Linked Data & Enrico (OU) & Mainly related to CHILD, MEETUPS, and WP4 \\\hline
MIDI LD & https://midi-ld.github.io & Linked Data & Albert (KCL) & Basis for WP2, useful for WP3. Large RDF KG and ontology of linked MIDI file contents from 500K MIDIs from the Web \\\hline
SPARQL Anything & https://github.com/SPARQL-Anything/sparql.anything & Software library, CLI & Enrico Daga (OU) & Support tool to re-engineer non-RDF resources into Linked Data. Supports CSV, JSON, XML, HTML, ... \\\hline
RAMOSE & https://github.com/opencitations/ramose & Software & Marilena (UNIBO) & Python API manager on top of SPARQL endpoints \\\hline
Lucinda & https://github.com/opencitations/lucinda & Software & Marilena (UNIBO) & RDF browser based on JSON templates \\\hline
OSCAR & https://github.com/opencitations/oscar & Software & Marilena (UNIBO) & RDF Search engine based on JSON templates \\\hline
MusoW & https://musow.kmi.open.ac.uk/ & Linked Data & Enrico (OU) & LOD registry of MH on the web to be expanded / enriched \\\hline
ArCo & https://w3id.org/arco & Linked Data / software & Val (UNIBO) & Ontology network and LOD / XML2RDF for normative italian Cultural Heritage XSD \\\hline
Lizard & https://github.com/anuzzolese/lizard & Software & Val (UNIBO) & Automatic generation of ontology-based APIs for querying knowledge graphs \\\hline
Neuma & http://neuma.huma-num.fr & Platform: library/dataset & Raphaël FS (CNAM) & WP1: Pilot FACETS 

MEI corpus, REST API, vizualisation of scores, analysis/annotations. \\\hline
Framester & https://github.com/framester/Framester & Linked Data / Software & Fiorela (UNIBO) & Frame-based ontological resource \\\hline
Edwin & https://github.com/luigi-asprino/edwin & Framework & Fiorela (UNIBO) & Builds and analyses Equivalence Set Graphs \\\hline
Cultural-ON & https://dati.beniculturali.it/cultural-ON/ENG.html & Linked Data & Fiorela (UNIBO) & Data on cultural institutes or sites \\\hline
Multilingual Corpus LBC & http://corpora.lessicobeniculturali.net/en/ & Dataset & Fiorela (UNIBO) & MusicBo \\\hline
Unicittà Corpus & Available in July 2021 & Dataset & Fiorela (UNIBO) & MusicBo \\\hline
CultuurLINK & https://cultuurlink.beeldengeluid.nl/ & Software & NISV & Open Source Tool for aligning vocabularies \\\hline
Spinque Desk & https://spinque.com/ & Software (commercial) & NISV/ Spinque & Integrate your data into a knowledge graph, design search solutions tailored to your needs and deploy them as APIs. \\\hline
FindLEr & https://github.com/enridaga/led-discovery & Software & Enrico (OU) & Supports the retrieval of listening experiences in digitized books. \\\hline
grlc & http://grlc.io https://github.com/CLARIAH/grlc & Software (OS) & Albert (KCL) & Supports automatic creation of KG APIs from shared SPARQL queries \\\hline
CLOVER & http://arco.istc.cnr.it:8081/ & Software & Val (UNIBO) & A prototype instance of OntoPortal for CH ontologies \\\hline
SPICE Linked Data Hub & http://spice.kmi.open.ac.uk 

http://github.com/mkdf/ & Software (OS) & Enrico (OU) & Example instance of a Linked Data Hub developed using the MK Data Factory suite. Supports data management (file repository, JSON streams), transformations to RDF (via SPARQL Anything) and publishing of SPARQL endpoints. \\\hline
\end{tabular}
\end{table}

\chapter{Technical Board}\label{ch:technical-board}
To support the collaborative development of the Pilots (Task 1.3), the project consortium established a Technical Board (TB), composed of representative members of the partners expecting to contribute on the development of the technical outputs of the project.
The role of the TB is to support the coordination of the technical activities and facilitate the interaction and collaboration.
In particular, the TB monitors technical developments fostering the sharing of expertise with the objective of maximising reuse of knowledge, skills, and resources.
Members of the TB are responsible of ensuring the quality of the technical outputs, both from the point of view of good practices of software engineering and from the perspective of providing guidance and support to users, developing and curating documentation, tutorials, and user guides.
In addition, the TB supervise the curation of resources for developers, so to maximise the reuse of the outputs by third-party organisations of the cultural heritage sector and industry. % \todo{More on this in the Methodology section?}
The TB establishes the methodology and tools for collaboration.
The present document provides details of the methodology and tools setup so far, which will be evaluated regularly and possibly changed to adapt to the concrete needs of the pilots.
Of particular importance is the dissemination of project outputs, which the TB aims at maximising by recommending the delivery of Open Source software published with a commercial-friendly Apache Licence 2.0, and asking consortium members to provide substantial arguments in case they require alternative policies. % \todo{More on this in the Methodology Section}

The project coordinator appointed Enrico Daga (OU) as Technical Director (TD), whose responsibility is the coordination of the Technical Board. 
The currently appointed members of the TB are listed in Table~\ref{tab:tbmembers}.% \todo{@All Please check and complete the table}
Partners are free to change the appointed person by communicating it to the Technical Director.
However, the TB follows an inclusive and open approach to discussion, inviting all developers, technologists, researchers, and interested people within the consortium to join TB meetings and participate. By default, TB meetings are open to all project members and we expect to organise closed meetings for exceptional reasons only.

\begin{table}[h]
    \caption{Technical Board appointed members}
    \label{tab:tbmembers}
    \centering
    \begin{tabular}{|p{5cm}|c|l|}\hline
\textbf{Member} &	\textbf{Partner} &	\textbf{Pilots} \\\hline
Enrico Daga & OU & CHILD, MEETUPS, ACCESS \\\hline
Johan Oomen & NISV & INTERLINK \\\hline
Raphaël Fournier-S'niehotta &	CNAM &	FACETS \\\hline
Mathieu d'Aquin & NUIG & \\\hline 
Rocco Tripodi &	UNIBO &	MUSICBO, INTERLINK, MEETUPS, BELLS	 \\\hline
Albert Meroño &	KCL & INTERLINK, FACETS \\\hline
Peter van Kranenburg &	KNAW &	ORGANS, TUNES \\\hline
Fiorela Ciroku & UNIBO & MUSICBO, INTERLINK, MEETUPS, BELLS \\\hline
Marilena Daquino &	UNIBO &	Web portal \\\hline
Thomas Bottini & IReMus CNRS & TONALITIES \\\hline
    \end{tabular}
\end{table}

% Scope, role and members 
% Coordinate technical activities / Facilitator 
% Share technical expertise 
% Maximize reuse 
% Ensure quality of outputs / documentation 
% Support dissemination of outputs (GH web site, documentation, tutorials) 
% Inc. Resources for developers 
% Members are … 
%\todo{Add what the TB did so far (see minutes) (Enrico)}